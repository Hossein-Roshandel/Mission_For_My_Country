\chapter{سرزمین قدیم و جدید ایران}
از ایام صباوت و دوران ولیعهدی که در سوئیس\index{سوئیس} تحصیل می‌کردم، واقعه‌ای در خاطرم مانده است. روزی شیرفروشی که هر بامداد گاری پر از ظروف شیر به دبیرستان میآورد، از من پرسید از کدام کشور به سوئیس آمده‌ام. گفتم از پرشیا\index{پرشیا} (ایران\index{ایران}) می‌آیم. گفت: آری، من پرشیا\index{پرشیا} را خوب می‌شناسم که یکی از شهرهای آمریکا\index{آمریکا} است. 
\par
سال‌ها ازا این واقعه گذشت. در این اواخر یکی از مستخدمین جوان دربار شاهنشاهی، مسافرتی به آمریکا\index{آمریکا} کرد. در هنگام بازگشت، واقعه‌ی عجیبی را که در آن کشور برای وی پیش آمده بود نقل کرد. او می‌گفت قبل از مسافرت به آمریکا\index{آمریکا}، همواره آرزو  داشتم که یکی از سرخ‌پوستان\index{سرخ‌پوست} آمریکایی\index{آمریکا} را به چشم ببینم و هنگامیکه به آن کشور رسیدم، این آرزو را با مهماندار آمریکایی\index{آمریکا} خود درمیان نهادم. این شخص گفت انجام این خواهش بسیار آسان است و چون مسافرت به یکی از نواحی که برای سکونت سرخ‌پوستان\index{سرخ‌پوست} معین شده جزو برنامه است، دیدار یک‌ نفر سرخپوست\index{سرخ‌پوست} میسر خواهد بود. 
هنگامیکه به ناحیه‌ی معهود رسیدیم، بسیار متأسف شدم زیرا سرخ‌پوستانی \index{سرخ‌پوست} که در آنحا بودند، آن سربندهای پرداری را که هالیوود\index{هالیوود} در فیلم‌های خود به دنیا عرضه می‌کند برسر نداشتند. بالاخره میهمان‌دار یک نفر سرخ‌پوست\index{سرخ‌پوست} را که به لباس بومیان ملبس و سربند پردار بر سر داشت و چهره را با رنگ‌های مختلف منقش کرده بود به من معرفی نمود و این شخص به زبان فصیح انگلیسی از من پرسید اهل کدام کشورید؟ گفتم از کشور دوردستی می‌آیم که پرشیا\index{پرشیا} یا ایران\index{ایران} نام دارد.  
\par
به‌مجرد شنیدن نام ایران\index{ایران}، چهره‌ي این سرخ‌پوست\index{سرخ‌پوست} از شادی شکفت و با زبان فارسی فصیحی گفت: «سلام علیکم، حال شما چطور است؟»
\par
من از این برخورد به حیرت افتادم، ولی به زودی دریافتم که این سرخ‌پوست\index{سرخ‌پوست} در جنگ‌‌جهانی\index{جنگ جهانی دوم} جزو ارتش آمریکایی\index{آمریکا} مأمور در خلیج فارس\index{خلیج فارس} بوده است که مقدار هفت میلیون تن مهمات و ذخائر از راه ایران به روسیه\index{روسیه} رسانده و شکست قوای هیتلر\index{هیتلر} و پیروزی متفقین\index{متفقین} را تسریع نمود. و آن سرخ‌پوست\index{سرخ‌پوست} نتنها خود زبان فارسی\index{فارسی} را فراگرفته، بلکه به عده‌ای از افراد عشیره‌ی خود یاد داده و از تمدن و فرهنگ باستانی ما نیز اطلاعاتی کسب کرده است.
\par 
ای‌کاش عده‌ی کثیری از مردم باخترزمین\index{باخترزمین}، به اندازه‌ی آن سرخ‌پوست\index{سرخ‌پوست} از کشور من اطلاعاتی داشتند و می‌دانستند که ایران \index{ایران}به پیشرفت‌های تمدن بشری چه خدماتی کرده و در آینده نیز چنانکه ایمان قطعی من است، چه خدماتی به فرهنگ و معارف جهانی انچام تواند داد. 
\par
گاهی که فکر می‌کنم چرا ایران\index{ایران} در میان کشورهای خاورمیانه\index{خاورمیانه} بهتر از این معروفیت ندارد، دچار حیرت می‌شوم. زیرا از هرچه بگذریم، ایران\index{ایران} سهمی بزرگ در تمدن خاورمیانه\index{خاورمیانه} داشته و  ثروت سرشاری از ذوق و هنر و ادب و فلسفه، به جهان غرب موهبت کرده است. و به همانگونه که ملت آمریکا\index{آمریکا}، امروز بوسیله‌ی برنامه‌ی اصل چهار، کمک‌های فنی به کشورهای دیگر می‌کند، ما از اوایل تاریخ، صادرکننده‌ی فرهنگ و هنر به جهان بشریت بوده‌ایم. 
\par
با وصف این باید گفت که در سالهای اخیر که کشورهای خاورمیانه\index{خاورمیانه}، در صحنه سیاست جهانی قسمت مهمی پیدا کرده‌اند، مردم گیتی در هر گوشه و کنار، نسبت به ایران\index{ایران} و کسب اطلاع نسبت به سرزمین ما و کشور کهنسال من بیشتر از پیش ابراز علاقه می‌کنند. 
\par
در نقشه‌های جغرافیایی عالم یا خاورمیانه\index{خاورمیانه}، ایران\index{ایران}، یا پرشیا\index{پرشیا}، بطور برجسته‌ای نمایان است. این کشور، از کشور آلاسکا\index{آلاسکا} بزرگتر و مساحت آن دوبرابر ایالت تگزاس\index{تگزاس} و از مجموع مساحت کشورهای فرانسه\index{فرانسه}، سوئیس\index{سوئیس}، ایتالیا\index{ایتالیا}، اسپانیا\index{اسپانیا}، پرتغال\index{پرتغال}، بلژیک\index{بلژیک} و لوکزامبورگ \index{لوکزامبورگ}بیشتر است. وضع جغرافیایی ما طوری است که هزاران سال، نقطه‌ی اتصال خطوط یا چهارراه گیتی بوده‌ایم. و این نکته همانقدر در روزگاری که مردم با کاروان‌ها مسافرت می‌کردند صادق بود، امروز نیز که قرن هواپیماهای جت و موشک‌های هدایت‌شونده است، صدق می‌کند. 
\par
جمعیت ایران\index{ایران}، به نسبت هر کیلومتر مربع، کم است ولی عده‌ی نفوس ما که بیست میلیون است، برابر جمعیت قاره‌مانند استرالیا\index{استرالیا} است.
تهران\index{تهران}، که پایتخت من است، یکی از شهرهایی است که در دنیا با سرعت توسعه پیدا کرده است؛ بطوریکه جمعیت آن از زمان جنگ دوم جهانی\index{جنگ جهانی دوم} (که پانصد هزار نفر جمعیت داشت) تا امروز سه برابر شده و به یک میلیون و پانصد هزار نفر بالغ گردیده است.
\par
البته یکی از علل ازدیاد نفوس تهران\index{تهران} آن است که همانطور که در بسیاری از کشورهای گیتی پیش آمده، عده‌ی کثیری از مردم، مساکن اصلی خویش را گذاشته و در شهر توطن اختیار کرده‌اند؛ ولی روی‌هم‌رفته، جمعیت پایتخت ما به سرعت رو به فزونی رفته است. 







 

