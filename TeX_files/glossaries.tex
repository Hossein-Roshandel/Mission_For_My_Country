%\newglossaryentry{sample1}{name={sample1},description={first example}}
%\newglossaryentry{sample2}{name={sample2},description={second example}}
\newglossaryentry{شاه طهماسب اول}{name={شاه طهماسب اول},description={(۱۴ فوریه ۱۵۱۴ – ۱۴ مه ۱۵۷۶) پسر ارشد شاه اسماعیل یکم و دومین پادشاه از دودمان صفویان بود}}

\newglossaryentry{شاه طهماسب دوم}{name={شاه طهماسب دوم},description={دهمین پادشاه صفوی ایران بین سال ۱۱۰۱ تا ۱۱۱۱ هجری شمسی بود. او پس از اشغال اصفهان به‌دست افغان‌ها و کشته شدن شاه سلطان حسین، برای مدتی بر بخشی از ایران حکومت می‌کرد و در پی شکست از عثمانی و بستن قرارداد صلح، توسط نادرشاه از سلطنت برکنار و کشته شد}}

\newglossaryentry{ناصرالدین‌شاه}{name={ناصرالدین‌شاه},description={(۲۵ تیر ۱۲۱۰ – ۱۲ اردیبهشت ۱۲۷۵) که پیش از دوران پادشاهی ناصرالدین میرزا خوانده می‌شد، معروف به «قبلهٔ عالم»، «سلطان صاحبقران» و بعد از کشته شدن توسط میرزا رضا کرمانی «شاهِ شهید»، چهارمین شاه از دودمان قاجار ایران بود. او با نزدیک به ۵۰ سال پادشاهی، پس از شاپور دوم ساسانی و تهماسب اول صفوی طولانی‌ترین دورهٔ پادشاهی در میان تمامی شاه‌های تاریخ ایران را داراست. او به افتخار نیم قرن سلطنت بر ایران، خود را صاحبقران نامید. او همچنین نخستین پادشاه ایرانی بود که خاطرات خود را نوشت}}

\newglossaryentry{مصدق}{name={مصدق},description={مشهور به دکتر مصدق و ملقب به مصدق‌السلطنه، سیاستمدار، حقوقدان، نمایندهٔ هشت دوره مجلس شورای ملی، استاندار، وزیر و دو دوره نخست‌وزیر ایران بود.}}

\newglossaryentry{راه ابریشم}{name={راه ابریشم},description={جادهٔ ابریشم یا راه ابریشم شبکهٔ راه‌های به‌هم‌پیوسته‌ای با هدف بازرگانی در آسیا بود که خاور و باختر و جنوب آسیا را به هم و به شمال آفریقا و خاور اروپا پیوند می‌داد؛ مسیری که تا سدهٔ پانزدهم میلادی به‌مدت ۱٬۷۰۰ سال، بزرگ‌ترین شبکهٔ بازرگانی دنیا بود.}}%