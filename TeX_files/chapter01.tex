\chapter{سرزمین قدیم و جدید ایران}

از ایام صباوت و دوران ولیعهدی که در سوئیس\index{سوئیس} تحصیل می‌کردم، واقعه‌ای در خاطرم مانده است. روزی شیرفروشی که هر بامداد گاری پر از ظروف شیر به دبیرستان میآورد، از من پرسید از کدام کشور به سوئیس آمده‌ام. گفتم از پرشیا\index{پرشیا} (ایران\index{ایران}) می‌آیم. گفت: آری، من پرشیا\index{پرشیا} را خوب می‌شناسم که یکی از شهرهای آمریکا\index{آمریکا} است. 
\par
سال‌ها ازا این واقعه گذشت. در این اواخر یکی از مستخدمین جوان دربار شاهنشاهی، مسافرتی به آمریکا\index{آمریکا} کرد. در هنگام بازگشت، واقعه‌ی عجیبی را که در آن کشور برای وی پیش آمده بود نقل کرد. او می‌گفت قبل از مسافرت به آمریکا\index{آمریکا}، همواره آرزو  داشتم که یکی از سرخ‌پوستان\index{سرخ‌پوست} آمریکایی\index{آمریکا} را به چشم ببینم و هنگامیکه به آن کشور رسیدم، این آرزو را با مهماندار آمریکایی\index{آمریکا} خود درمیان نهادم. این شخص گفت انجام این خواهش بسیار آسان است و چون مسافرت به یکی از نواحی که برای سکونت سرخ‌پوستان\index{سرخ‌پوست} معین شده جزو برنامه است، دیدار یک‌ نفر سرخپوست\index{سرخ‌پوست} میسر خواهد بود. 
هنگامیکه به ناحیه‌ی معهود رسیدیم، بسیار متأسف شدم زیرا سرخ‌پوستانی \index{سرخ‌پوست} که در آنحا بودند، آن سربندهای پرداری را که هالیوود\index{هالیوود} در فیلم‌های خود به دنیا عرضه می‌کند برسر نداشتند. بالاخره میهمان‌دار یک نفر سرخ‌پوست\index{سرخ‌پوست} را که به لباس بومیان ملبس و سربند پردار بر سر داشت و چهره را با رنگ‌های مختلف منقش کرده بود به من معرفی نمود و این شخص به زبان فصیح انگلیسی از من پرسید اهل کدام کشورید؟ گفتم از کشور دوردستی می‌آیم که پرشیا\index{پرشیا} یا ایران\index{ایران} نام دارد.  
\par
به‌مجرد شنیدن نام ایران\index{ایران}، چهره‌ي این سرخ‌پوست\index{سرخ‌پوست} از شادی شکفت و با زبان فارسی فصیحی گفت: «سلام علیکم، حال شما چطور است؟»
\par
من از این برخورد به حیرت افتادم، ولی به زودی دریافتم که این سرخ‌پوست\index{سرخ‌پوست} در جنگ‌‌جهانی\index{جنگ جهانی دوم} جزو ارتش آمریکایی\index{آمریکا} مأمور در خلیج فارس\index{خلیج فارس} بوده است که مقدار هفت میلیون تن مهمات و ذخائر از راه ایران به روسیه\index{روسیه} رسانده و شکست قوای هیتلر\index{هیتلر} و پیروزی متفقین\index{متفقین} را تسریع نمود. و آن سرخ‌پوست\index{سرخ‌پوست} نتنها خود زبان فارسی\index{فارسی} را فراگرفته، بلکه به عده‌ای از افراد عشیره‌ی خود یاد داده و از تمدن و فرهنگ باستانی ما نیز اطلاعاتی کسب کرده است.
\par 
ای‌کاش عده‌ی کثیری از مردم باخترزمین\index{باخترزمین}، به اندازه‌ی آن سرخ‌پوست\index{سرخ‌پوست} از کشور من اطلاعاتی داشتند و می‌دانستند که ایران \index{ایران}به پیشرفت‌های تمدن بشری چه خدماتی کرده و در آینده نیز چنانکه ایمان قطعی من است، چه خدماتی به فرهنگ و معارف جهانی انچام تواند داد. 
\par
گاهی که فکر می‌کنم چرا ایران\index{ایران} در میان کشورهای خاورمیانه\index{خاورمیانه} بهتر از این معروفیت ندارد، دچار حیرت می‌شوم. زیرا از هرچه بگذریم، ایران\index{ایران} سهمی بزرگ در تمدن خاورمیانه\index{خاورمیانه} داشته و  ثروت سرشاری از ذوق و هنر و ادب و فلسفه، به جهان غرب موهبت کرده است. و به همانگونه که ملت آمریکا\index{آمریکا}، امروز بوسیله‌ی برنامه‌ی اصل چهار، کمک‌های فنی به کشورهای دیگر می‌کند، ما از اوایل تاریخ، صادرکننده‌ی فرهنگ و هنر به جهان بشریت بوده‌ایم. 
\par
با وصف این باید گفت که در سالهای اخیر که کشورهای خاورمیانه\index{خاورمیانه}، در صحنه سیاست جهانی قسمت مهمی پیدا کرده‌اند، مردم گیتی در هر گوشه و کنار، نسبت به ایران\index{ایران} و کسب اطلاع نسبت به سرزمین ما و کشور کهنسال من بیشتر از پیش ابراز علاقه می‌کنند. 
\par
در نقشه‌های جغرافیایی عالم یا خاورمیانه\index{خاورمیانه}، ایران\index{ایران}، یا پرشیا\index{پرشیا}، بطور برجسته‌ای نمایان است. این کشور، از کشور آلاسکا\index{آلاسکا} بزرگتر و مساحت آن دوبرابر ایالت تگزاس\index{تگزاس} و از مجموع مساحت کشورهای فرانسه\index{فرانسه}، سوئیس\index{سوئیس}، ایتالیا\index{ایتالیا}، اسپانیا\index{اسپانیا}، پرتغال\index{پرتغال}، بلژیک\index{بلژیک} و لوکزامبورگ \index{لوکزامبورگ}بیشتر است.
وضع جغرافیایی ما طوری است که هزاران سال، نقطه‌ی اتصال خطوط یا چهارراه گیتی بوده‌ایم. و این نکته همانقدر در روزگاری که مردم با کاروان‌ها مسافرت می‌کردند صادق بود، امروز نیز که قرن هواپیماهای جت و موشک‌های هدایت‌شونده است، صدق می‌کند. 
\par
جمعیت ایران\index{ایران}، به نسبت هر کیلومتر مربع، کم است ولی عده‌ی نفوس ما که بیست میلیون است، برابر جمعیت قاره‌مانند استرالیا\index{استرالیا} است.
تهران\index{تهران}، که پایتخت من است، یکی از شهرهایی است که در دنیا با سرعت توسعه پیدا کرده است؛ بطوریکه جمعیت آن از زمان جنگ دوم جهانی\index{جنگ جهانی دوم} (که پانصد هزار نفر جمعیت داشت) تا امروز سه برابر شده و به یک میلیون و پانصد هزار نفر بالغ گردیده است.
\par
البته یکی از علل ازدیاد نفوس تهران\index{تهران} آن است که همانطور که در بسیاری از کشورهای گیتی پیش آمده، عده‌ی کثیری از مردم، مساکن اصلی خویش را گذاشته و در شهر توطن اختیار کرده‌اند؛ ولی روی‌هم‌رفته، جمعیت پایتخت ما به سرعت رو به فزونی رفته است.
\par
قسمت بزرگی از کشور، خشک و بی‌آب‌وعلف است ولی در قسمت‌های دیگر مقدار باران سالیانه بسیار زیاد، و از جنگل‌های انبوه و مزارع برنج پوشیده است. آب قسمت‌‌های نسبتاً خشک کشور از ذوب آب برف‌های کوهستان‌هاست که مانند حلقه‌ای، گرداگرد قلات مرکزی، ایران\index{ایران} را احاطه کرده و غالب سلسله‌هایی هم با یکدیگر تقاطع می‌کنند، جز در ناحیه‌ی کویر، در سایر نقاط ایران\index{ایران} نقطه‌ای نیست که از کوهستان‌ها فاصله‌ی بسیار داشته باشد. آبی را که از کوهستان‌ها جاری است بوسیله‌ی حفر قنات به مزارع و قراء می‌رسانند و اخیراً اقدام به حفر چاه‌های عمیق کرده‌ایم و متخصصین می‌گویند که برای رفع نیازمندی‌های جمعیتی که سه برابر جمعیت فعلی ایران\index{ایران} باشد، در ایران\index{ایران} آب به قدر کافی وجود دارد.
\par
تفاوت آب‌و‌هوا در ایران\index{ایران} زیاد است و در نقاط مختلف و فصول سال فرق می‌کند، هرچند به گمان من باید همین اختلاف آب‌و‌هوا را یکی از نیروهای مؤثر شمرد.
مردم کشور ما دارای نیروی بدنی فوق‌العاده هستند و شاید این مسأله از تصادفات نباشد که می‌بینیم مردم ایران در ورزش‌هایی نظیر وزنه‌برداری و کشتی بسیار قوی هستند و در مسابقه‌های بین‌المللی، ورزشکاران ما در این رشته‌ها بیش از حد تناسب جمعیت کشور به دریافت جوایز قهرمانی نائل گشته‌اند.
\par
از لحاظ هوش و سرعت انتقال، مردم ایران\index{ایران}، چنانکه ذکر خواهد شد، شهرتی بسزا دارند و با توسعه و تعمیم تعلیمات اجباری در سراسر کشور به نظر من می‌توان امیدوار بود که در آینده نیز ایرانیان در علم، هنر، کشاورزی، صنعت، و بازرگانی سهم مؤثر و ذیقیمتی خواهند داشت.
\par
از لحاظ معادن، ایران\index{ایران} دارای منابع گرانبها است. البته از حیث نفت\index{نفت}، ما یکی از تولیدکنندگان بزرگ جهان محسوب می‌شویم  و از این لحاظ شهرت یافته‌ایم. در خاورمیانه\index{خاورمیانه}، صنعت نفت\index{نفت} از ایران\index{ایران} آغاز گردیده و طبق بررسی‌ها و تحقیقات علمی اخیر، کشور من مانند کشتی بزرگی است که روی دریای نفت\index{نفت} قرار گرفته باشد.
آنچه برای دیگران نسبتاً مجهول مانده این است که ما دارای معادن گرانبها و ذی‌قیمت دیگر مخصوصاً ذغال‌سنگ، آهن، مس، مانگانز، کروم، احجار کریمه و بسیاری از املاح شیمیایی مانند بورات، سولفات و نمک طبرزد، هستیم که به مقادیر زیاد قابل‌صدور وجود دارد و هنوز استخراج و بهره‌برداری از آنها در مراحل مقدماتی است.  
\par
خاک ایران\index{ایران} تقریباً درتمام نقاط جز در ناحیه‌ی کویر بزرگ نمک که مانند کشور استرالیا\index{استرالیا} قسمتی بزرگ از فلات مرکزی ایران\index{ایران} را فرا گرفته، حاصلخیز است.
هرجا آب به زمین قابل زراعت برسد، انواع مختلف محصول مانند گندم، جو، ذرت، برنج، پنبه، سیب‌زمینی، ماش، یونجه، چغندرقند، نیشکر، تنباکو  و چای زراعت می‌شود و سبزی‌های خوردنی مانند کلم، شلغم، پیاز، بادنجان، خیار و غیره به‌عمل می‌آید.  
\par
کسانیکه به ایران\index{ایران} آمده‌اند، از میوه‌های پرآب و معطر آن، مخصوصاً از سیب، هلو، زردآلو، انگور، گیلاس، آلو، گلابی، انار و مرکبات گوناگون از لیمو، نارنگی، خرما، و زیتون لذت برده‌اند. انواع خربزه، پسته، فندق، و بادام در کشور می‌روید. تربیت مواشی در ایران\index{ایران} رونق بسزایی دارد و روستائیان و افراد عشایر ما، که لباس قبیله مخصوص به خویش را می‌پوشند، در کوهستان‌ها و جلگه‌های کشور به پرورش اغنام و سایر مواشی مشغولند. 
در چند سال اخیر، مکانیزه‌کردن کشاورزی آغاز شده است و در فصول بعد برنامه‌ای را که برای تقسیم املاک میان روستائیان و کشاورزان داریم، و اجرای آن آغاز گشته است، شرح خواهم داد. 
\par
ما معمولاً مقدار نسبتاً قلیلی موااد غذایی وارد و مقدار معتنابهی صادر می‌کنیم و مخصوصاً صادرات میوه، پسته و بادام ما زیاد است. قسمت عمده‌ی خاویاری که در بازارهای جهان به اسم خاویار روسی به فروش می‌رسد، از ایران \index{ایران} می‌آید که به روسیه\index{روسیه} و آمریکا\index{آمریکا} و سایر کشورها صادر می‌شود (یکی از ظرفا گفته است که ماهی خاویار از نظر تفاوت عقیده از شمال بحر خزر\index{دریای خزر} به سواحل جنوبی آن مهاجرت کرده است. ولی حق این است که سالها پیش از انقلاب روسیه، این ماهی آب‌های گرم، سواحل جنوب بحر خزر\index{دریای خزر}، یعنی سواحل ایران\index{ایران}، را برای توطن و تولید و تکثیر مرجح شناخته بود).
شیلات ما دارای منابع بزرگی است که هنوز دست‌نخورده است. 
صفحه ۱۵




 

