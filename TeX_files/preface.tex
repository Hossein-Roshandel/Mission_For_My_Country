\chapter{دیباچه}
\paragraph{
این کتاب برای انجام منظوری که سالیان دراز ضرورت آن احساس می‌شد نگارش یافته است. تا آنجا که اطلاع دارم، از بیست و پنج قرن پیش که شاهنشاهی ایران بنیان‌گذاری شده است، من نخستین شاهنشاهی هستم که شرح زندگانی خود را بطور مرتب و با تسلسل تاریخی تألیف و تدوین کرده‌ام.
}
\paragraph{
البته در قرن شانزدهم میلادی، یعنی دوهزار سال پس از آغاز شاهنشاهی ایران، \gls{شاه طهماسب اول} اول که یکی از سلاطین این کشور بود، تاریخچه‌‌ی مختصر زندگانی خویش را به رشته‌ی تحریر درآورد و دویست سال بعد، در دوره‌ی سلطنت \gls{شاه طهماسب دوم}، یک نفر کشیش فرانسوی، شرح احوال مجملی به نام این پادشاه انتشار داد، به این منظور که ثابت کند وی فرزند یک نفر فرانسوی بوده است. 
}
\paragraph{
در قرن نوزدهم نیز، \gls{ناصرالدین‌شاه}، سفرنامه‌ی دل‌چسبی درباره‌ی مسافرت خویش به اروپا و مشاهدات و توجه خود به فهم رموز ترقیات دول باختری نگاشت، ولی هیچ یک از سران تاجدار کشور من، شرح زندگانی خویش را با روش مرتب و منظمی مدون نساختند. این بود که تقریباً پس از چهارده سال سلطنت، مصمم شدم که این کار نخستین بار به دست من انجام پذیرد. 
غرض تنها آن نبود که در میان شاهنشاهان این کشور در نوشتن شرح احوال پیشقدم باشم، بلکه احساس کردم که نگارش چنین کتابی برای آشنایی به گذشته و راهنمایی آینده‌ی ایران بسیار ضروری است.
}
\paragraph{
در قرن اخیر، ما که در خاورمیانه زندگانی می‌کنیم، در فهم ارزش حقیقی خویش کوتاهی کرده و اغلب در تهیه‌ی نقشه‌ها و برنامه‌های صحیح برای ترقیات آینده‌ی کشورهای خود به غفلت گذرانیده‌ایم. گاهی از آنچه در کشور انجام یافته اطلاعات مبهمی داشته‌ایم ولی در تشخیص موانع تشکیلاتی و غیر آن، که مانع پیشرفت بوده است، قصور کرده و در تعیین هدف‌ها و آمال و مقاصدی که برای آینده‌ی ما ضروری است، به اصل مسامحه توسل جسته‌ایم. به این جهات، به نظر من موقع آن بود که شاهنشاه ایران، این نقیصه‌ی اساسی را جبران کند. 
}
\paragraph{
نگارش این کتاب در سال ۱۳۳۷ آغاز و در اپاخر سال ۱۳۳۹ پایان پذیرفت. در طول این مدت، هرچند وظائف خطیر دیگر اوقات مرا به خود مشغول می‌داشت، اما هفته‌ای نمی‌گذشت که جز در ایام مسافرت، ساعتی چند به نگارش این کتاب مصروف نشود و حتی در سفرها نیز فکر تنظیم مطالب مندرج در آن از خاطر من بیرون نمی‌رفت. 
}
\paragraph{
	کتابی که اینک انتشار پیدا میکند، با سایر کتب مربوط به شرح احوال یا خاطرات تفاوت بسیار دار، زیرا این کتاب در واقع تنها شرح زندگانی من نیست، بلکه تاریخ احوال یک کشوری است. 
}
\paragraph{
	فصل اول این کتاب به تاریخچه‌ی مختصر میراث شگفت‌انگیز و سابقه‌ی درخشان تاریخی کشور ایران اختصاص یافته و در فصل بعد ذکر خدمات شگرف و عقیده و استنباط من درباره‌ی شخصیت پدرم، که در تحولات اخیر خاورمیانه از برجسته‌ترین افراد بود، پرداخته شده است. آنگاه به شرح دوران کودکی و دوره‌ی تحصیلات من در اروپا و توجه و مراقبت مخصوصی که پدرم در تربیت من داشت و مرا برای تعهد مسئولیت‌های سنگین کنونی آماده می‌فرمود می‌پردازد، و پس از آن بحرانی که در اثر جنگ جهانگیر دوم پیش آمد، و کشور ایران مورد تجاوز قرار گرفت، و من در سن بیست‌و‌یک‌سالگی بجای پدر وظائف سلطنت را بر عهده گرفتم، شرح داده می‌شود.
} 
\paragraph{
	در این کتاب، اطلاعات خود را درباره‌ی قضایای آن سالها که شخصی به نام \gls{مصدق}، کشور ایران را به طریقی که مخصوص خودش بود می‌گردانید و صنعت نفت ما به حال وقفه در آمد و اقتصاد ما فلج شده و مشعل آزادی از نور و فروغ افتاده و تقریباً خاموش گشته بود، شرح می‌دهم و خواهم گفت که چگونه آزادی را دوباره بدست آوردیم و نیز توضیح خواهم داد که چطور در اثر آن تحارب تلخ در من عقیده‌ی ناسیونالیزم مثبت به وجود آمد.  
}
\paragraph{
	در این کتاب، شمه‌ای از اصطکاک تمدن باختر با ایمان و امیدواری در ایجاد یک نحوه‌ی ارتباط جدید بین شرق و غرب سخن می‌رود و حدود و میزان توسعه و پیشرفت‌های اقتصادی کشور، و عقیده‌ی من درباره‌ی مراحل سه‌گانه‌ی دموکراسی، و اقداماتی که برای ایجاد دموکراسی حقیقی در این کشور به‌عمل آمده، ذکر می‌شود. در مسأله‌ی اصلاحات مالکیت ارضی و سایر قدم‌هایی که برای کمک به کشاورزانی که در پنج‌هزار قریه‌ی ایران زندگانی می‌کنند، برداشته شده و از وظایفی که زنان ما در ایران کنونی دارند به اختصار سخن به میان می‌آید. و عقیده‌ی کلی من درباره‌ی آموزش‌و‌پرورش در ایران شرح داده می‌شود. 
	و مسأله‌ی نفت ایران از لحاظ سیاسی و اقتصادی و فنی مورد بحث قرار می‌گیرد. وموقعیت سوق‌الحیشی ایران در خاورمیانه و مکنونات قلبی من در شرایطی که برای استقرار صلح و آرامش در این بخش از جهان و در سایر نقاط گیتی ضروری است بیان خواهد شد. و بالاخره در این کتاب، به طور اختصار از طرز زندگانی و کار شبانه‌روزی خود و وظیفه‌ی عملی که در این کشور تاریخی برعهده‌ی مقام سلطنت است، سخن خواهد رفت.   
}
\paragraph{
در طی تمام فصول این کتاب، کوشش من همواره آن بوده است که در ذکر موانع و مشکلات، تنها به اجمال و بطور اشاره اکتفا نکنم. مثلاً مسئله‌ی درستکاری افراد، چه در سازمان‌های دولتی و چه در دستگاه‌های شخصی و حرفه‌ای و اجتماعات مختلف، همچنان از مسائل دشوار ما است. 
دستگاه های اداری کشوری، باآنکه در سالهای اخیر از هر حیث خیلی بهتر از سالهای پیش است، باز در بسیاری از موارد کهنه و فرسوده است و بیسوادی و فقر و بیماری هنوز در کشور ما ریشه‌کن نگشته است. اما باید به خاطر آورد که ما می‌خواهیم پیشرفت‌هایی را که در ممالک مترقی پس از چندین نسل و حتی چندین قرن به مرحله‌ی ایحاد رسیده است، در ظرف چند سال بوجود آوریم.   
}
\paragraph{
	امروز در تمام نقاط گیتی، نسبت به خاورمیانه، که کانون تضادها است، ابراز توجه و علاقه می‌شود، زیرا از یک طرف این ناحیه در تمدن جهانی سهمی بزرگ داشته و از طرف دیگر همواره مهد حوادث و کانون تشنجات بوده است. به عقیده‌ی من، اوضاع برای تجدید حیات خاورمیانه مساعد است و دلیلی ندارد که ایران، چنانکه بارها به شهادت تاریخ موجد اینگونه تحولات بوده است، بار دیگر برای کمک به ایجاد یک چنین تحولی ناتوان باشد.   
}